%定義

\documentclass[10pt,twocolumn]{jarticle} 
\usepackage{algorithmic, algorithm}
\usepackage[dvipdfmx]{graphicx}
\usepackage[dvipdfmx]{hyperref}
\usepackage{fancybox}
\usepackage{comment}
\usepackage{amsmath}
\usepackage{amssymb}
\usepackage{amsfonts}
\usepackage{euler}


\pagestyle{empty}

%余白とか

\setlength{\topmargin}{-3.0cm} 
\setlength{\textheight}{28.0cm} 
\setlength{\textwidth}{18.5cm}
\setlength{\oddsidemargin}{-1.3cm} 
\setlength{\columnsep}{.5cm}
\newcommand{\noin}{\noindent}
\catcode`@=\active \def@{\hspace{0.9bp}-\hspace{0.9bp}}
\newtheorem{dfn}{定義}[section]


%タイトル

\title{第100回研究報告 \\ Turing Award, G\"{o}del Prize を受賞する為に必要な最低限の知識について}
\setcounter{footnote}{1}
\author{神奈川 太郎\if0\thanks{神奈川大学大学院理学研究科情報科学専攻 田中研究室}\fi}
\date{\today}
\西暦

%タイトル作成

\begin{document}

\maketitle
\thispagestyle{empty}

% 1.何について話すか(概要) => 2.具体例(図だけで十分) => 3.結論(一番伝えたい事)
%
% 結論に向けて,流れを意識して書く.
%
% 構文.主語,述語,目的語をはっきり書く.
%
% 受動態を使わない.
%
% 辞書,参考書,先生に伺うことより,適切な単語,表現で文章を作成する.
%
% 内容を削る.必要最小限
%
% コンマを列挙以外で2つ以上使う,``~の~の'',の如きは使わない.

% 

\section{概略}
\noindent ゼミ資料の内容を数行で書く.どんな疑問について,どんな所に着眼して,どんな検討をし,どんな結論を得たのか.

\section{準備}
\noindent 先生の授業の様に前回の復習から始める.予備知識を復習する.``連とは何か''など.

\section{前回までの経緯,問題点}
\noindent 何が問題となっていたかを概説する.

\section{本論}
\subsection{表の挿入}
加減乗除を理解している必要がある.表を用いて確かめるとか確かめないとか.

\begin{table}[htbp]
 \begin{center}
  \caption{加減乗除が分かるようになるかもしれないルールリスト}
%  {\small
  \begin{tabular}{ccccccc}
        & &             & &        & &             \\ \cline{1-3} \cline{5-7}
 Filter & & $F_{1}$      & & Filter & & $F_{1}$      \\ \cline{1-3} \cline{5-7}
 $R_{1}$ & & $*$ $0$ $*$ $1$ & & $R_{7}$ & & $*$ $*$ $1$ $0$ \\
 $R_{2}$ & & $0$ $0$ $0$ $0$ & & $R_{8}$ & & $0$ $1$ $*$ $*$ \\
 $R_{3}$ & & $0$ $*$ $0$ $0$ & & $R_{9}$ & & $*$ $1$ $1$ $*$ \\
 $R_{4}$ & & $0$ $*$ $1$ $*$ & & $R_{10}$ & & $*$ $0$ $0$ $0$ \\
 $R_{5}$ & & $1$ $1$ $0$ $0$ & & $R_{11}$ & & $*$ $1$ $*$ $1$ \\
 $R_{6}$ & & $*$ $0$ $1$ $*$ & & $R_{12}$ & & $*$ $*$ $*$ $1$ \\ \cline{1-3} \cline{5-7}
  \end{tabular}
%  }
  \label{rulelist}
 \end{center}
\end{table}

\subsection{図の挿入}
図\ref{paper_rbtrie}には,一ヶ所誤りがある.見つけよ.見つけられれば,$1$から$2$までの数を数えられている.\par
\noindent 下の用に記述すると,
\begin{verbatim}
\begin{figure}[!htbp]
 \centering{
  \scalebox{0.8}{\input{rbtrie.tps}}
  \caption{表\ref{rulelist}から構成したRun-Based Trie}
  \label{paper_rbtrie}
 }
\end{figure}
\end{verbatim}

図\ref{paper_rbtrie}が適当な位置に挿入される.
\begin{figure*}[!htbp]
 \centering{
  \scalebox{0.9}{\input{rbtrie.tps}}
  \caption{表\ref{rulelist}から構成したRun-Based Trie}
  \label{paper_rbtrie}
 }
\end{figure*}

\subsection{数式}
数式モードにはいくつか方法がある.
\begin{itemize}
 \item \verb|$ $|で挟んで文章内に数式を入れる
 \item \verb|\equation|環境を用いる(数式に番号を振る)
 \item \verb|\[ \]|を用いる(数式に番号を振らない)
\end{itemize}

\subsubsection{Sub Graph Mergingに関する諸命題}

ルール$R_{k}$の部分グラフを$S_{k}$とする.部分グラフ$S_{k}$に含まれるルールの評価パケット数の平均を$A(S_{k})$と表す.ルール$R_{j}$がルール$R_{i}$に従属することを$R_{i} \Downarrow R_{j}$と表す.

\begin{equation}
\forall \ R_{j} \in S_{k} \ (\lnot (\exists R_{i} (R_{i} \Downarrow R_{j})) \Rightarrow \| R_{j} \| \leq A(S_{k}))
\end{equation}

部分グラフ$A,B$の位数がそれぞれ$n, m$であるとし,部分グラフの各ルールをルール番号順(若しくはポリシー違反をおこさない順)に$a_{i}, b_{i} \ (i \leq m)$と表す.
\begin{align}
 \begin{aligned}
  & \frac{\sum_{k=1}^{n}\|a_{k}\|}{n} = \frac{\sum_{k=1}^{m}\|b_{k}\|}{m} \\
  & \\
  & \Rightarrow \\
  & \\
  & \sum_{k=1}^{n}k \cdot \|a_{k}\| + \sum_{k=n+1}^{n+m}k \cdot \|b_{k-n}\| \\
= &\sum_{k=1}^{m}k \cdot \|b_{k}\| + \sum_{k=m+1}^{m+n}k \cdot \|a_{k-m}\|
 \label{nothing}
 \end{aligned}
\end{align}

上の命題の結論部分の右辺第2項と左辺第2項をそれぞれ次のように式変形する.
\begin{align}
 \begin{aligned}
  \sum_{k=n+1}^{n+m}k \cdot \|b_{k-n}\| &=
  \sum_{n+1 \leq k \leq n+m}k \cdot \|b_{k-n}\| \\
  &= \sum_{n+1 \leq k+n \leq n+m}(k+n) \cdot \|b_{(k+n)-n}\| \\
  &= \sum_{1 \leq k \leq m}(k+n) \cdot \|b_{k}\| \\
  &= \sum_{k=1}^{m}(k+n) \cdot \|b_{k}\|
 \end{aligned}
\end{align}

\begin{align}
 \begin{aligned}
   \sum_{k=m+1}^{m+n}k \cdot \|a_{k-m}\| &= \sum_{m+1 \leq k \leq m+n}k \cdot \|a_{k-m}\| \\
   &= \sum_{m+1 \leq (k+m) \leq m+n}(k+m) \cdot \|a_{(k+m)-m}\| \\
   &= \sum_{1 \leq k \leq n}(k+m) \cdot \|a_{k}\| \\
   &= \sum_{k=1}^{n}(k+m) \cdot \|a_{k}\|
 \end{aligned}
\end{align}
これより,\ref{nothing}の結論部分の命題は
\begin{align}
 \begin{aligned}
  & \sum_{k=1}^{n}k \cdot \|a_{k}\| + \sum_{k=n+1}^{n+m}k \cdot \|b_{k-n}\| \\
= &\sum_{k=1}^{m}k \cdot \|b_{k}\| + \sum_{k=m+1}^{m+n}k \cdot \|a_{k-m}\| \\
\iff %& \\
  & \sum_{k=1}^{n}k \cdot \|a_{k}\| + \sum_{k=1}^{m}(k+n) \cdot \|b_{k}\| \\
= &\sum_{k=1}^{m}k \cdot \|b_{k}\| + \sum_{k=1}^{n}(k+m) \cdot \|a_{k}\| \\
\iff %& \\
  & \sum_{k=1}^{n}(k+m) \cdot \|a_{k}\| -\sum_{k=1}^{n}k \cdot \|a_{k}\| \\
 = & \sum_{k=1}^{m}(k+n) \cdot \|b_{k}\| - \sum_{k=1}^{m}k \cdot \|b_{k}\| \\
\iff %&\\
  &\sum_{k=1}^{n} \bigl((k+m) \cdot \|a_{k}\| - k \cdot \|a_{k}\| \bigr) \\
 = &\sum_{k=1}^{m} \bigl((k+n) \cdot \|b_{k}\| - k \cdot \|b_{k}\| \bigr) \\
\iff %&\\
  & \sum_{k=1}^{n} m \cdot \|a_{k}\| = \sum_{k=1}^{m} n \cdot \|b_{k}\| \\
\iff & m \sum_{k=1}^{n} \|a_{k}\| = n \sum_{k=1}^{m} \|b_{k}\| \\
\iff & \frac{\sum_{k=1}^{n} \|a_{k}\|}{n} = \frac{\sum_{k=1}^{m} \|b_{k}\|}{m}
 \end{aligned}
\end{align}
同値変形により得られた命題は,\ref{nothing}で仮定した条件に等しいので,命題\ref{nothing}は正しい.つまり,部分グラフ$A, B$の平均の評価パケット数が同じで$V(A) \cap V(B) = \phi$のとき,どちらの部分グラフのルールをルールリストの上位に配置しても遅延$L(R)$の値は同じである.

田中研究室では,パケットの頻度分布を$F$,ルールリストを${\mathbf R}$として,遅延$L(F, {\mathbf R})$を次のように定義する.

\[
 L(F, {\mathbf R}) = \sum_{i=1}^{n-1} i \times \| R_{i}(F, {\mathbf R})\| + (n-1) \times \| R_{n}(F, {\mathbf R})\| 
\]

$\|R_{i}(F, {\mathbf R}\|$は,パケットの頻度分布$F$,ルールリスト${\mathbf R}$における,$R_{i}$の評価パケット数を表す.$n$は,ルールリスト中のルールの数である.

\par

\verb|\equation|の例を下記に示す.\par
(\ref{easy})の方程式を解け.
\begin{equation}
 1 + x = 2
 \label{easy}
\end{equation}

自然数の全体がなす集合は,
\begin{equation}
\mathbb{N} \cong \mathbb{N} + \mathbf{1}
 \label{nat}
\end{equation}
を満たすような最小の$\mathbb{N}$である.

\section{PDFの挿入}
%\begin{figure}[H]
 \includegraphics[width=.1\textwidth, bb=0 0 841.92 595.32, clip]{pdftest.pdf}
%\end{figure}

\section{擬似コード}
\begin{algorithm}[H]
\caption{: cutRunFromRule($R_{i}$)}
\label{cut_run}
\begin{algorithmic}[1]
 \STATE $j \leftarrow 1$ // Run number
 \STATE $k \leftarrow 0$ // iterator for rule string
 \STATE $L \leftarrow R_{i}.string.length()$ // iterator for rule string
% \STATE $run.start$ // Run start point
 \STATE $sign \leftarrow false$
 %\STATE $term \leftarrow false$
 \STATE $run.string \leftarrow$ ""
 \WHILE{$k < L$}
  \IF{$R_{i}.string[k] \neq $ '$*$'} %// [k] is $0$ or $1$
   \IF {$k = 0 \ \lor \ R_{i}.string[k-1] = $'$*$'}
    \STATE $run.start \leftarrow k + 1$
   \ENDIF
   \STATE $run.string$ += $R_{i}.string[k]$
   \STATE $sign = true$
  \ELSE                              %// [k] is $*$
   \IF {$sign = true$}
    \STATE $R_{i}.hasRun.push\_back(run)$
    \STATE $run.string \leftarrow$ ""
    \STATE $sign = false$
    \STATE $j \leftarrow j + 1$
   \ENDIF
  \ENDIF
  \STATE $k \leftarrow k + 1$
 \ENDWHILE
 \IF {$sign = true$}
  \STATE $R_{i}.hasRun.push\_back(run)$
 \ENDIF
\STATE addTerminalMark($R_{i}.hasRun$) 
\end{algorithmic}
\end{algorithm}

\section{まとめ,今後の課題}
因数分解を理解する予定.

\par

{\small
 \bibliographystyle{ieice.bst}
 \bibliography{template}
}

\section{チェックリスト}
\begin{itemize}
\item $5 + 3 = ?$
\item $5 \times 5 = ?$
\end{itemize}

\appendix
\section{参考文献の書き方}

参考文献の書く為には,\verb|makefile|中の \verb|pbibtex|行のコメントアウト(\verb|#|)を外し,本文中参照すれば良い.例えば,texファイル中に\verb|\cite{2014RbtHARADA}|(\verb|2014RbtHARADA|は,\verb|tamplate.bib|中で論文\cite{2014RbtHARADA}を参照する為に対応付けたラベルである)と書けば,\par
{\centering

\cite{2014RbtHARADA}

}
\noindent の様に参考文献に対応する番号を表示する.また,

\begin{verbatim}
{\small
\bibliographystyle{ieice.bst}
\bibliography{template}
}
\end{verbatim}
をtexファイル中に書いた場所に参考文献が表示される.但し,\verb|pbibtex| を行う(\verb|makefile|中のコメントアウトを取り除く)のに,本文中に上記の\verb|\bibliographystyle{~}|を記さない,または,本文中で参照(\verb|\cite{~}|)を行わない,ということをすると,コンパイルエラーになる(この\verb|makefile|,若しくはtexファイルが悪いだけで,良い方法があるかもしれないので,解決法をご存知の方は,教えて下さい).\par
\noindent \verb|r201470039hs at kanagawa-u.ac.jp|


\end{document}
