%定義

\documentclass[10pt,twocolumn]{jarticle} 
\usepackage{graphicx}
\usepackage{fancybox}
\usepackage{comment}
\usepackage{amsmath}
\usepackage{amssymb}
\usepackage{amsfonts}
\usepackage{euler}

\pagestyle{empty}

%余白とか

\setlength{\topmargin}{-3.0cm} 
\setlength{\textheight}{28.0cm} 
\setlength{\textwidth}{18.5cm}
\setlength{\oddsidemargin}{-1.3cm} 
\setlength{\columnsep}{.5cm}
\newcommand{\noin}{\noindent}
\catcode`@=\active \def@{\hspace{0.9bp}-\hspace{0.9bp}}
\newtheorem{dfn}{定義}[section]


%タイトル

\title{第100回研究報告 \\ Turing Award, G\"{o}del Prize を受賞する為に必要な最低限の知識について}
\setcounter{footnote}{1}
\author{神奈川 太郎\if0\thanks{神奈川大学大学院理学研究科情報科学専攻 田中研究室}\fi}
\date{\today}
\西暦

%タイトル作成

\begin{document}

\maketitle
\thispagestyle{empty}

% 1.何について話すか(概要) => 2.具体例(図だけで十分) => 3.結論(一番伝えたい事)
%
% 結論に向けて,流れを意識して書く.
%
% 構文.主語,述語,目的語をはっきり書く.
%
% 受動態を使わない.
%
% 辞書,参考書,先生に伺うことより,適切な単語,表現で文章を作成する.
%
% 内容を削る.必要最小限
%
% コンマを列挙以外で2つ以上使う,``~の~の'',の如きは使わない.

% 

\section{概略}
\noindent ゼミ資料の内容を数行で書く.どんな疑問について,どんな所に着眼して,どんな検討をし,どんな結論を得たのか.

\section{準備}
\noindent 先生の授業の様に前回の復習から始める.予備知識を復習する.``連とは何か''など.

\section{前回までの経緯,問題点}
\noindent 何が問題となっていたかを概説する.

\section{本論}
加減乗除を理解している必要がある.

\section{まとめ,今後の課題}
因数分解を理解する予定.

\par

{\small
 \bibliographystyle{ieice.bst}
 \bibliography{template}
}

\section{チェックリスト}
\begin{itemize}
\item $5 + 3 = ?$
\item $5 \times 5 = ?$
\end{itemize}

\appendix
\section{参考文献の書き方}

参考文献の書く為には,\verb|makefile|中の \verb|pbibtex|行のコメントアウト(\verb|#|)を外し,本文中参照すれば良い.例えば,texファイル中に\verb|\cite{siftsearch}|(\verb|siftsearch|は,\verb|tamplate.bib|中で論文\cite{siftsearch}を参照する為に対応付けたラベルである)と書けば,\par
{\centering

\cite{siftsearch}

}
\noindent の様に参考文献に対応する番号を表示する.また,

\begin{verbatim}
{\small
\bibliographystyle{ieice.bst}
\bibliography{template}
}
\end{verbatim}
をtexファイル中に書いた場所に参考文献が表示される.但し,\verb|pbibtex| を行う(\verb|makefile|中のコメントアウトを取り除く)のに,本文中に上記の\verb|\bibliographystyle{~}|を記さない,または,本文中で参照(\verb|\cite{~}|)を行わない,ということをすると,コンパイルエラーになる(この\verb|makefile|,若しくはtexファイルが悪いだけで,良い方法があるかもしれないので,解決法をご存知の方は,教えて下さい).\par
\noindent \verb|r201470039hs at kanagawa-u.ac.jp|


\end{document}
