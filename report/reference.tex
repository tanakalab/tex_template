%定義

\documentclass[10pt,twocolumn]{jarticle} 
\usepackage{graphicx}
\usepackage{fancybox}
\usepackage{comment}
\usepackage{amsmath}
\usepackage{amssymb}
\usepackage{amsfonts}
\usepackage{euler}

\pagestyle{empty}

%余白とか

\setlength{\topmargin}{-3.0cm} 
\setlength{\textheight}{28.0cm} 
\setlength{\textwidth}{18.5cm}
\setlength{\oddsidemargin}{-1.3cm} 
\setlength{\columnsep}{.5cm}
\newcommand{\noin}{\noindent}
\catcode`@=\active \def@{\hspace{0.9bp}-\hspace{0.9bp}}
\newtheorem{dfn}{定義}[section]


%タイトル

\title{template.bib に記述している文献}
\setcounter{footnote}{1}
\author{参考 次郎\if0\thanks{神奈川大学大学院理学研究科情報科学専攻 田中研究室}\fi}
\date{\today}
\西暦

%タイトル作成

\begin{document}

\maketitle
\thispagestyle{empty}

% 1.何について話すか(概要) => 2.具体例(図だけで十分) => 3.結論(一番伝えたい事)
%
% 結論に向けて,流れを意識して書く.
%
% 構文.主語,述語,目的語をはっきり書く.
%
% 受動態を使わない.
%
% 辞書,参考書,先生に伺うことより,適切な単語,表現で文章を作成する.
%
% 内容を削る.必要最小限
%
% コンマを列挙以外で2つ以上使う,``~の~の'',の如きは使わない.

% 

template.bib には,下記の論文のBibTeXが記述されている.\cite{survey_Taylor_2005},\cite{hicuts19},\cite{rbt},\cite{yuto},\cite{hypercuts},\cite{2014RbtHARADA},\cite{conf/infocom/HamedEA06},\cite{hikin},\cite{hicuts20},\cite{quad-trie},\cite{siftsearch},\cite{Taylor_ClassBench},\cite{takeyama},\cite{sgm},\cite{grouper-conf},\cite{pagiamtzis-jssc:2006}

\bibliographystyle{ieice.bst}
\bibliography{template}


\end{document}
