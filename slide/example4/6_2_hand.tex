%\documentclass[12pt,dvipdfmx,mathserif,uplatex,aspectratio=32]{beamer}
\documentclass[a4paper,10pt]{jarticle}
\usepackage{beamerarticle}
%\usetheme{Singapore}
\usetheme{CambridgeUK}
%\AtBeginShipoutFirst{\special{pdf:tounicode EUC-UCS2}}
\usepackage{euler}
\usepackage{multicol}
\usepackage{subfigure}
\usepackage{graphicx}
\usepackage{tikz}
\usetikzlibrary{intersections, calc}
\usetikzlibrary{decorations.pathmorphing}
\renewcommand{\kanjifamilydefault}{\gtdefault}
\setbeamerfont{title}{size=\LARGE}
\setbeamertemplate{navigation symbols}{} % ナビゲーションをなくす
\setbeamertemplate{items}[default]

\makeatletter
\newcommand{\figcaption}[1]{\def\@captype{figure}\caption{#1}}
\newcommand{\tblcaption}[1]{\def\@captype{table}\caption{#1}}
\makeatother

%% \myCoilArrow{長さ} : スライド用の右矢印
\newcommand{\myRightArrow}[1]{%
\tikz[baseline=-1pt]
  \draw (0,0)--(.5,0)--(.5,-0.1)--(.7,.1)--(.5,.3)--(.5,.2)--(.5,.2)--(0,.2)--(0,0);
}

\title{Aggregated Bit-Vector (ABV) \\ Cross Producting}
\subtitle{2015年度 前期輪講 \\ "Survey and Taxonomy of Packet Classification Techniques" \\ Abstract and Introduction}
\author{原田崇司}
\date[]{2015年6月16日}
\institute{神奈川大学大学院 理学研究科 情報科学専攻 田中研究室}
% [..]に省略名が書ける

\begin{document}



% 1ページ目
\frame{\maketitle}

\section*{目次}

% 2ページ目
%\begin{frame}{目次}
%  \tableofcontents
%\end{frame}



\section{Aggregated Bit-Bector (2001)}

% 3ページ目
\begin{frame}{ABV前に提案されたフィルタリング法の問題点}

 \begin{itemize}
  \item 特殊なハードを使用 \hspace{5mm} (TCAM) \vspace{3mm}
  \item フィールド数$2$以下で構成されるルールに特化 \\ (CrossProducting, Hicuts, RFC, etc) \\ \vspace{2mm} \myRightArrow \\ フィールド数$3$以上では使用メモリ量が膨大 %\vspace{3mm}
 \end{itemize}

\end{frame}

% 4ページ目
\begin{frame}{Aggregated Bit Vector}

\vspace{-10mm}

\begin{figure}[h]
 \def\@captype{table}
 \begin{minipage}[t]{.3\textwidth}
  {\footnotesize
  {\centering
  %
 \begin{tabular}{c|l|l}
  Rule    & $Field_{1}$ & $Field_{2}$ \\ \hline
  $F_{0}$ & $00*$       & $00*$ \\ \hline
  $F_{1}$ & $00*$       & $01*$ \\ \hline
  $F_{2}$ & $10*$       & $11*$ \\ \hline
  $F_{3}$ & $11*$       & $10*$ \\ \hline
  $F_{4}$ & $0*$        & $10*$ \\ \hline
  $F_{5}$ & $0*$        & $11*$ \\ \hline
  $F_{6}$ & $0*$        & $0*$  \\ \hline
  $F_{7}$ & $1*$        & $01*$ \\ \hline
  $F_{8}$ & $1*$        & $0*$  \\ \hline
  $F_{9}$ & $11*$       & $0*$  \\ \hline
  $F_{10}$ & $10*$       & $10*$
 \end{tabular}

  }
  }
  \end{minipage}
  %
  \hfill
  %
  \begin{minipage}[c]{.6\textwidth}
   \vspace{8mm}
   \scalebox{0.45}{\input{abv.tps}}
  \end{minipage}
\end{figure}

{\centering
四角で囲われたのがBit Vector,その下の太字がABV

}

\end{frame}

% 5ページ目
\begin{frame}{Aggregated Bit Vector}

\begin{block}{$H_{1} = 00 \dots, \ H_{2} = 11 \dots$のパケットをBVで探索}

\begin{equation*}
 \begin{array}{ll}
  Field_{1} & 11001110000 \\ 
  Field_{2} & 00100100000 \\ \hline
  AND      & 00000100000 
 \end{array}
\end{equation*}
\end{block}

\begin{block}{$H_{1} = 00 \dots , \ H_{2} = 11 \dots$をABVも用いて探索}
\vspace{-5mm}
\begin{figure}[h]
 \def\@captype{table}
 \begin{minipage}[t]{.45\textwidth}
\begin{equation*}
 \begin{array}{ll}
  Field_{1} & \mathbf{110} \\
  Field_{2} & \mathbf{110} \\ \hline
  AND      &  \mathbf{110}
 \end{array}
\end{equation*}

  \end{minipage}
  %
  \hfill
  %
  \begin{minipage}[t]{.45\textwidth}
\begin{equation*}
 \begin{array}{ll}
  Field_{1} & 1100111 \\
  Field_{2} & 0010010 \\ \hline
  AND      &  0000010
 \end{array}
\end{equation*}

  \end{minipage}
\end{figure}

初めにABVのANDをとって,BVの探索する場所を絞り込む

\end{block}
\end{frame}

% 6ページ目
\begin{frame}{Aggregated Bit Vector}

\noindent ABVを用いた場合に残念なことが起こる例

\begin{figure}[h]
 \def\@captype{table}
 \begin{minipage}[t]{.4\textwidth}
  {\footnotesize
  {\centering
  %
  \begin{tabular}{c|l|l} 
   Filter   & $Field_{1}$ & $Field_{2}$ \\ \hline
   $F_{0} $ & $00000*$    & $11*$   \\ \hline
   $F_{1} $ & $1*$        & $1010*$  \\ \hline
   $F_{2} $ & $00000*$    & $0*$    \\ \hline
   $F_{3} $ & $01*$       & $1010*$  \\ \hline
   $F_{4} $ & $00000*$    & $100*$   \\ \hline
   $F_{5} $ & $100*$      & $1010*$  \\ \hline
   $F_{6} $ & $00000*$    & $000*$  \\ \hline
   $F_{7} $ & $001*$      & $1010*$  \\ \hline
   $F_{8} $ & $00000*$    & $01*$   \\ \hline
   $F_{9} $ & $11*$      & $1010*$  \\ \hline
   $F_{10} $ & $0000*$    & $0*$   \\ \hline
   $F_{11} $ & $010$      & $1010*$  \\ \hline
   $F_{12} $ & $0000*$    & $1010*$   
  \end{tabular}
  %\tblcaption{RuleList1}

  }
  }
  \end{minipage}
  %
  \hfill
  %
  \begin{minipage}[c]{.55\textwidth}
  Aggrgeation Size $= 2$
  \par
  \vspace{5mm}

  $H_{1} = 00000 \dots, \ H_{2} = 1010 \dots $

   \begin{equation*}
    \begin{array}{lll}
     F_{1} & 1111111 & ({\rm AVB \ of \ } 00000) \\
     F_{2} & 1111111 & ({\rm AVB \ of \ } 1010) \\ \hline
           & 1111111 & 
    \end{array}
   \end{equation*}

   \begin{equation*}
    \begin{array}{lll}
     F_{1} & 1010101010101 & ({\rm VB \ of \ } 00000) \\
     F_{2} & 0101010101011 & ({\rm VB \ of \ } 1010) \\ \hline
           & 0000000000001 &
    \end{array}
   \end{equation*}

  \end{minipage}
\end{figure}

\end{frame}


% 7ページ目
\begin{frame}{Aggregated Bit Vector}

ポリシーに違反しないようにフィルタを並び替える

\begin{figure}[h]
 \def\@captype{table}
 \begin{minipage}[t]{.4\textwidth}
  {\footnotesize
  {\centering
  %
  \begin{tabular}{c|l|l} 
   Filter   & $Field_{1}$ & $Field_{2}$ \\ \hline
   $F_{0} $ & $00000*$    & $11*$   \\ \hline
   $F_{1} $ & $1*$        & $1010*$  \\ \hline
   $F_{2} $ & $00000*$    & $0*$    \\ \hline
   $F_{3} $ & $01*$       & $1010*$  \\ \hline
   $F_{4} $ & $00000*$    & $100*$   \\ \hline
   $F_{5} $ & $100*$      & $1010*$  \\ \hline
   $F_{6} $ & $00000*$    & $000*$  \\ \hline
   $F_{7} $ & $001*$      & $1010*$  \\ \hline
   $F_{8} $ & $00000*$    & $01*$   \\ \hline
   $F_{19} $ & $11*$      & $1010*$  \\ \hline
   $F_{10} $ & $0000*$    & $0*$   \\ \hline
   $F_{11} $ & $010$      & $1010*$  \\ \hline
   $F_{12} $ & $0000*$    & $1010*$   
  \end{tabular}
  %\tblcaption{RuleList1}

  }
  }
  \end{minipage}
  %
  \hfill
  %
  \begin{minipage}[c]{.55\textwidth}
  {\footnotesize
  {\centering
  %
  \begin{tabular}{c|l|l} 
   Filter   & $Field_{1}$ & $Field_{2}$ \\ \hline
   $F_{0}^{'} $ & $00000*$    & $11*$   \\ \hline
   $F_{1}^{'} $ & $00000*$    & $0*$    \\ \hline
   $F_{2}^{'} $ & $00000*$    & $100*$   \\ \hline
   $F_{3}^{'} $ & $00000*$    & $000*$  \\ \hline
   $F_{4}^{'} $ & $00000*$    & $01*$   \\ \hline
   $F_{5}^{'} $ & $0000*$    & $0*$   \\ \hline
   $F_{6}^{'} $ & $0000*$    & $1010*$   \\ \hline
   $F_{7}^{'} $ & $1*$        & $1010*$  \\ \hline
   $F_{8}^{'} $ & $01*$       & $1010*$  \\ \hline
   $F_{9}^{'} $ & $100*$      & $1010*$  \\ \hline
   $F_{10}^{'} $ & $001*$      & $1010*$  \\ \hline
   $F_{11}^{'} $ & $11*$      & $1010*$  \\ \hline
   $F_{12}^{'} $ & $010$      & $1010*$ 
  \end{tabular}
  %\tblcaption{RuleList1}

  }
  }
  \end{minipage}
\end{figure}

\end{frame}

% 8ページ目
\begin{frame}{Aggregated Bit Vector}
\begin{figure}[h]
 \def\@captype{table}
 \begin{minipage}[c]{.5\textwidth}
   \begin{equation*}
    \begin{array}{lll}
     F_{1} & 1111111000000 & ({\rm VB \ of \ } 00000) \\
     F_{2} & 0000001111111 & ({\rm VB \ of \ } 1010) \\ \hline
           & 0000001000000 &
    \end{array}
   \end{equation*}

   \begin{equation*}
    \begin{array}{lll}
     F_{1} & 1111000 & ({\rm AVB \ of \ } 00000) \\
     F_{2} & 0001111 & ({\rm AVB \ of \ } 1010) \\ \hline
           & 0001000 & 
    \end{array}
   \end{equation*}

   \begin{equation*}
    \begin{array}{lll}
     F_{1} & 10 & ({\rm search \ only } \ 6,7 \ {\rm bits}) \\
     F_{2} & 11 & ({\rm search \ only } \ 6,7 \ {\rm bits}) \\ \hline
           & 10 & 
    \end{array}
   \end{equation*}

  \end{minipage}
  %
  \hfill
  %
  \begin{minipage}[c]{.43\textwidth}
  
  \vspace{3mm}
  ルールを並び替えた後に$H_{1} = 00000 \dots$, $\ H_{2} = 1010 \dots $ を探索

  \[
   7 + 13 = 20 \to 7 + 2 = 9
  \]
  メモリアクセス数減少
  \end{minipage}
\end{figure}

\end{frame}

% 9ページ目

\section{Cross Producting (1998)}

% 10ページ目
\begin{frame}{Set Pruning Tree}

\begin{figure}[h]
 \def\@captype{table}
 \begin{minipage}[t]{.3\textwidth}
  {\scriptsize
  {\centering
  %
  \begin{tabular}{c|c|c} 
   Filter   & Destination & Source \\ \hline
   $F_{1} $ & $0*$        & $10*$   \\ \hline
   $F_{2} $ & $0*$        & $01*$   \\ \hline
   $F_{3} $ & $0*$        & $1*$   \\ \hline
   $F_{4} $ & $00*$       & $1*$   \\ \hline
   $F_{5} $ & $00*$       & $11*$   \\ \hline
   $F_{6} $ & $10*$       & $1*$   \\ \hline
   $F_{7} $ & $*$         & $00*$   
  \end{tabular}
  \tblcaption{RuleList1}

  }
  }
  \end{minipage}
  %
  \hfill
  %
  \begin{minipage}[c]{.6\textwidth}
   \vspace{8mm}
   \scalebox{0.45}{\input{set_pruning_tree.tps}}
   \vspace{-6mm}
   \caption{Set Pruning Tree of RuleList1}
  \end{minipage}
\end{figure}

\vspace{3mm}

Dest-Trieの白丸ノードは,ルートからそのノードへのパスで構成されるビット列が,Destination フィールドにないことを表現

\end{frame}

% 10ページ目
\begin{frame}{Set Pruning Tree}

\begin{figure}[h]
 \def\@captype{table}
 \begin{minipage}[c]{.55\textwidth}
   \scalebox{0.45}{\input{set_pruning_tree_search.tps}}
  \end{minipage}
  %
  \hfill
  %
  \begin{minipage}[c]{.3\textwidth}
  $D = 00 \dots$
  \par
  $S = 10 \dots$
  \end{minipage}
\end{figure}

\vspace{5mm}
Dest-Trie を $0 \to 0$,Source-Trie を $1 \to 0$ と辿り,$F_{1}$を返す.
\par
\vspace{3mm}
(Longest Prefix Matchingなので,途中の$F_{3}, F_{4}$は無視)
\end{frame}

% 11ページ目
\begin{frame}{Grid of Tries}

 \begin{itemize}
  \item 二つのフィールドしか持たないルール限定の方法 \vspace{3mm}
  \item フィールドはプレフィックスで指定 \\ (レンジルールはプレフィックスルールへ変換) \vspace{3mm}
  \item パケットとルールのマッチングは Longest Prefix Matching
 \end{itemize}

\end{frame}

% 12ページ目
\begin{frame}{Grid of Tries}

\begin{figure}[h]
 \def\@captype{table}
 \begin{minipage}[t]{.3\textwidth}
  {\scriptsize
  {\centering
  %
  \begin{tabular}{c|c|c} 
   Filter   & Destination & Source \\ \hline
   $F_{1} $ & $0*$        & $10*$   \\ \hline
   $F_{2} $ & $0*$        & $01*$   \\ \hline
   $F_{3} $ & $0*$        & $1*$   \\ \hline
   $F_{4} $ & $00*$       & $1*$   \\ \hline
   $F_{5} $ & $00*$       & $11*$   \\ \hline
   $F_{6} $ & $10*$       & $1*$   \\ \hline
   $F_{7} $ & $*$         & $00*$   
  \end{tabular}
  \tblcaption{RuleList1}

  }
  }
  \end{minipage}
  %
  \hfill
  %
  \begin{minipage}[c]{.6\textwidth}
   \vspace{8mm}
   \scalebox{0.45}{\input{grid_of_trie.tps}}
   \vspace{-6mm}
   \caption{Grid of Tries of RuleList1}
  \end{minipage}
\end{figure}

\vspace{3mm}

ルールの重複を避けるために,\par Destinationに完全に一致する箇所にのみSourceのトライを構成

\end{frame}

% ページ目
\begin{frame}{Grid of Tries}

\begin{figure}[h]
 \def\@captype{table}
 \begin{minipage}[c]{.55\textwidth}
   \scalebox{0.55}{\input{grid_of_trie_search.tps}}
  \end{minipage}
  %
  \hfill
  %
  \begin{minipage}[c]{.2\textwidth}
  $D = 001 \dots$
  \par
  $S = 001 \dots$
  \end{minipage}
\end{figure}

\vspace{3mm}
D-Trieを$0 \to 0$と辿るが,S-Trieを辿れず,D-Trieの$0*$へ
\par
\vspace{2mm}
S-Trieを$0$と辿るがフィルタに合致しないので,D-Trieの$*$へ
\par
\vspace{2mm}
S-Trieを$0 \to 0$と辿って,$F_{7}$を返す.
\end{frame}

% ページ目
\begin{frame}{Grid of Tries (with Switch Pointers)}

 \begin{figure}
  {\centering
  \scalebox{0.6}{\input{grid_of_trie_switch.tps}}

  }
 \end{figure}

\vspace{5mm}

探索時間計算量を$O(dW^{2})$から$O(dW)$とするために
\par
\vspace{3mm}
スイッチポインタを与える

\end{frame}

% ページ目
\begin{frame}{Extend Grid of Tries}

\begin{figure}[h]
 \def\@captype{table}
 \begin{minipage}[t]{.5\textwidth}
  {\scriptsize
  {\centering
  %
  \begin{tabular}{c|l|l|c|c|c} 
   Filter   & DA      & SA     & DP     & SP     & Prot \\ \hline
   $F_{1} $ & $0*$    & $10*$  & $*$    & $80$   & TCP  \\ \hline
   $F_{2} $ & $0*$    & $01*$  & $*$    & $80$   & TCP  \\ \hline
   $F_{3} $ & $0*$    & $1*$   & $17$   & $17$   & UDP \\ \hline
   $F_{4} $ & $00*$   & $1*$   & $*$    & $*$    & $*$ \\ \hline
   $F_{5} $ & $00*$   & $11*$  & $*$    & $*$    & TCP \\ \hline
   $F_{6} $ & $10*$   & $1*$   & $17$   & $17$   & UDP \\ \hline
   $F_{7} $ & $*$     & $00*$  & $*$    & $*$    & $*$
   %% $F_{8} $ & $0$     & $10*$  & $*$    & $100$  & TCP \\ \hline
   %% $F_{9} $ & $0*$    & $1*$   & $17$   & $44$   & UDP \\ \hline
   %% $F_{10}$ & $0*$    & $10*$  & $80$   & $*$    & TCP \\ \hline
   %% $F_{11}$ & $111*$  & $000*$ & $*$    & $44$   & UDP 
  \end{tabular}
  \tblcaption{ポート番号,プロトコル追加}

  }
  }
  \end{minipage}
  %
  \hfill
  %
  \begin{minipage}[c]{.48\textwidth}
   \begin{itemize}
    \item Port番号
    \item プロトコル
   \end{itemize}
   もフィルタリングの基準として使用

\par
(ポート番号は範囲指定不可)
\par
\vspace{5mm}
ポート番号の四つ組合わせに \par 対してハッシュ表を作成
\par
\vspace{2mm}
(DA, SA)の組 =
\par  \{ ($*$, $*$), ($*$, 指定), (指定, $*$), (指定, 指定) \} 
  \end{minipage}
\end{figure}

\end{frame}

% ページ目
\begin{frame}{Extend Grid of Tries(未完)}

 \begin{figure}
  {\centering
  \scalebox{0.6}{\input{extended_grid_of_trie.tps}}

  }
 \end{figure}

\end{frame}


\begin{frame}{Cross Producting}

\begin{figure}[h]
 \def\@captype{table}
 \begin{minipage}[t]{.5\textwidth}
  {\scriptsize
  {\centering
  %
  \begin{tabular}{c|l|l|c|c|c} 
   Filter   & DA      & SA     & DP     & SP     & Prot \\ \hline
   $F_{1} $ & $0*$    & $10*$  & $*$    & $80$   & TCP  \\ \hline
   $F_{2} $ & $0*$    & $01*$  & $*$    & $80$   & TCP  \\ \hline
   $F_{3} $ & $0*$    & $1*$   & $17$   & $17$   & UDP \\ \hline
   $F_{4} $ & $00*$   & $1*$   & $*$    & $*$    & $*$ \\ \hline
   $F_{5} $ & $00*$   & $11*$  & $*$    & $*$    & TCP \\ \hline
   $F_{6} $ & $10*$   & $1*$   & $17$   & $17$   & UDP \\ \hline
   $F_{7} $ & $*$     & $00*$  & $*$    & $*$    & $*$
   %% $F_{8} $ & $0$     & $10*$  & $*$    & $100$  & TCP \\ \hline
   %% $F_{9} $ & $0*$    & $1*$   & $17$   & $44$   & UDP \\ \hline
   %% $F_{10}$ & $0*$    & $10*$  & $80$   & $*$    & TCP \\ \hline
   %% $F_{11}$ & $111*$  & $000*$ & $*$    & $44$   & UDP 
  \end{tabular}
  %\tblcaption{ポート番号,プロトコル追加}

  }
  }
  \end{minipage}
  %
  \hfill
  %
  \begin{minipage}[c]{.48\textwidth}
  {\footnotesize
  {\centering
  %
  \begin{tabular}{llccc} 
   DA    & SA     & DP   & SP    & Prot \\ 
   $0*$  & $10*$  & $*$  & $80$  & TCP  \\ 
   $00*$ & $01*$  & $17$ & $17$  & UDP \\ 
   $10*$ & $1*$   &      & $*$   & $*$ \\ 
   $*$   & $11*$  &      &       &     \\
         & $00*$  &      &       &
   %% $F_{8} $ & $0$     & $10*$  & $*$    & $100$  & TCP \\ \hline
   %% $F_{9} $ & $0*$    & $1*$   & $17$   & $44$   & UDP \\ \hline
   %% $F_{10}$ & $0*$    & $10*$  & $80$   & $*$    & TCP \\ \hline
   %% $F_{11}$ & $111*$  & $000*$ & $*$    & $44$   & UDP 
  \end{tabular}
  %\tblcaption{ポート番号,プロトコル追加}

  }
  }

  \end{minipage}
 
  \vspace{8mm}
  各フィールドにおいて異なるルールを集めてその直積をとる.
  \par
  \vspace{3mm}
  そして,各々の組み合わせに対して最優先ルールを与える.
\end{figure}



\end{frame}



\begin{frame}{On Demand Cross-Producting}
Cross-Productingは,空間計算量が$O(N^{K})$となり実用的でない.
($N$はルール数,$K$はフィールド数)

\vspace{5mm}

\myRightArrow \par フィルタリングを行いながら Cross-Producting表を作成

\end{frame}

\begin{frame}{参考文献}

{\scriptsize
\begin{thebibliography}{99}
\setlength{\itemsep}{-.5zw}
\beamertemplatetextbibitems

\bibitem{ABV} F. Baboescu, and G. Varghese, “Scalable Packet Classification,” SIGCOMM Comput. Commun. Rev., vol.31, no.4, pp.199–210, Aug. 2001. 
\vspace{5mm}

\bibitem{CrossProducting} V. Srinivasan, G. Varghese, S. Suri, and M. Waldvogel, “Fast and Scalable Layer Four Switching,” SIGCOMM Comput. Commun. Rev., vol.28, no.4, pp.191–202, Oct. 1998.

\end{thebibliography}
}

\end{frame}


\end{document}
