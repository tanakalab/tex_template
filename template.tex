%定義

\documentclass[10pt,twocolumn]{jarticle} 
\usepackage{graphicx}
\usepackage{fancybox}
\usepackage{comment}
\usepackage{amsmath}
\usepackage{amssymb}
\usepackage{amsfonts}
\usepackage{euler}

\pagestyle{empty}

%余白とか

\setlength{\topmargin}{-3.0cm} 
\setlength{\textheight}{28.0cm} 
\setlength{\textwidth}{18.5cm}
\setlength{\oddsidemargin}{-1.3cm} 
\setlength{\columnsep}{.5cm}
\newcommand{\noin}{\noindent}
\catcode`@=\active \def@{\hspace{0.9bp}-\hspace{0.9bp}}
\newtheorem{dfn}{定義}[section]


%タイトル

\title{題名 一目で分からせることのできる題名}
\setcounter{footnote}{1}
\author{原田 崇司\if0\thanks{神奈川大学大学院理学研究科情報科学専攻 田中研究室}\fi}
\date{\today}
\西暦

%タイトル作成

\begin{document}

\maketitle
\thispagestyle{empty}

% 1. 何について話すか(概要) => 2. 具体例(図だけで十分) => 3. 結論(一番伝えたい事)
%
% 結論に向けて, 流れを意識して書く.
%
% 構文. 主語, 述語, 目的語をはっきり書く.
%
% 受動態を使わない.
%
% 辞書, 参考書, 先生に伺うことより, 適切な単語, 表現で文章を作成する.
%
% 内容を削る. 必要最小限
%
% コンマを列挙以外で2つ以上使う, ``~の~の'', の如きは使わない.

\section{概略}
\noindent ゼミ資料の内容を数行で書く. どんな疑問について, どんな所に着眼して, どんな検討をし, どんな結論を得たのか.

\section{準備}
\noindent 先生の授業の様に前回の復習から始める. 予備知識を復習する. ``連とは何か''など.

\section{前回までの経緯, 問題点}
\noindent 何が問題となっていたかを概説する.

\section{本論}


\section{まとめ, 今後の課題}

{\small
\bibliographystyle{../../../ieice.bst}
\bibliography{../../../packet_filtering}
}

\section{チェックリスト}
\begin{itemize}
\item 
\item 
\end{itemize}

\end{document}
